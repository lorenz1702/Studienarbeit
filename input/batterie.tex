%Fragen an Lehmann
%Soll ich noch Literatur dazu ein binden?
%Welchen Controller könnte ich verwenden?

\chapter{Batterie}\label{introduction}
Eine essenzielle Komponente bei der Realisierung eines individuell konstruierten E-Bikes ist die Batterie, welche den Motor mit der benötigten Energie versorgt. Im vorliegenden Abschnitt wird detailliert auf die Fertigung einer individuellen Lithium-Ionen-Batterie eingegangen, die durch die Auswahl passenden Zellen und ihre spezifische Konfiguration realisiert wird.\\

%Was muss die Batterie Leisten?
Die Batterie muss den spezifizierten Anforderungen hinsichtlich Leistung, Form und Stabilität entsprechen. In Bezug auf die Leistung ist eine Spannung von 48 Volt erforderlich, um den Motor zu versorgen, und die Batterie muss eine Leistung von 1500 Watt bereitstellen, um die gewünschte Performance zu gewährleisten. Dies wird durch diese Formel berechnet\ref*{eq:wattformel}:\\
\begin{equation}
    P = U \cdot I \longrightarrow I = \frac{P}{U}= \frac{1500W}{48V}= 31.25
    \label{eq:wattformel}
\end{equation}
\label{introduction2}
%Welche Kapazität soll die Batterie haben?

Die Batterie sollte nicht nur die erforderliche Leistung liefern, sondern auch eine stabile und sichere Form aufweisen. Die physikalische Stabilität der Batterie ist entscheidend, um sicherzustellen, dass sie den Belastungen während des Betriebs standhält und keine strukturellen Schwächen aufweist. Dies ist besonders wichtig, um mögliche Gefahren wie Leckagen oder mechanische Schäden zu vermeiden.\\

Darüber hinaus ist die Form der Batterie relevant, da sie in das E-Bike integriert werden muss. Die Form sollte daher kompatibel mit dem vorgesehenen Platzierungsort sein, um eine effiziente Nutzung des verfügbaren Raums zu ermöglichen. Die Batterie sollte sich leicht und sicher im vorgesehenen Bereich befestigen lassen, um eine optimale Integration in das Gesamtsystem zu gewährleisten.\\

Die Konstruktion der Batterie stellt einen maßgeblichen Schritt im Bau eines selbstgefertigten E-Bikes dar, da sie maßgeblich die Leistung und Reichweite des Fahrzeugs beeinflusst.


\section{Zellen}
%Welche Rollen spielen die Zellen im Batterie bau?
Die Zellen in einer Lithium-Ionen-Batterie für E-Bikes spielen eine entscheidende Rolle bei der Speicherung und Bereitstellung elektrischer Energie. Jede Zelle fungiert als eigenständige Energieeinheit, die miteinander kombiniert wird, um die gewünschte Spannung und Kapazität für den E-Bike-Motor zu erreichen. Ihre Aufgaben umfassen die Speicherung von Elektronen während des Ladevorgangs und die Abgabe dieser Elektronen während des Entladevorgangs, um den Motor mit Strom zu versorgen.\\

%Wie werden sie verbunden?
Die Zellen werden in der Batterie durch spezifische Verbindungsmethoden miteinander verbunden. Die Verbindung erfolgt durch das serielle Schalten der Zellen, wodurch die Einzelspannungen addiert werden. Dieser Schritt ermöglicht es, die für den E-Bike-Motor erforderliche höhere Gesamtspannung von 48 Volt zu erreichen. Die meisten Zellen haben 3.6 Volt Nennspannung um auf die 48 Volt zu kommen müssen 13 in Reihe geschaltet werden.\\

%Wie ist der schaltplan?
Die Entscheidung, Zellen in Serie oder parallel zu schalten, beeinflusst maßgeblich die Leistung und Charakteristik der Batterie. Die Reihenschaltung erhöht die Gesamtspannung, während die Parallelschaltung die Gesamtkapazität steigert. Dies hat direkte Auswirkungen auf die Leistungsfähigkeit des E-Bikes. Eine geeignete Kombination von Reihen- und Parallelschaltungen kann die erforderliche Spannung und Kapazität optimieren und somit die Fahrleistung und Reichweite des E-Bikes verbessern.\\

nsgesamt spielen die Zellen im Lithium-Ionen-Batteriebau für E-Bikes eine zentrale Rolle, und ihre Verbindung durch Reihen- und Parallelschaltungen ist entscheidend für die Leistungsfähigkeit und Effizienz des Batteriesystems.\\


\subsection{Art der Zellen}
Bei der Art der Zellen kommen nur zwei in Frage einmal die 21700-Zellen und 18650-Zellen\ref*{fig:1850VS21700}.

\begin{figure}[h]
    \centering
    \includegraphics[width=8cm]{images/18650-VS-21700.jpg}
    \caption{1850 VS 21700\cite{Tritek.12132021}}
    \label{fig:1850VS21700}
\end{figure}

Die Entscheidung zwischen 21700-Zellen und 18650-Zellen für die Batteriekonstruktion von E-Bikes erfordert eine eingehende wissenschaftliche Analyse verschiedener Faktoren. Einer der entscheidenden Aspekte ist die energetische Leistung und Kapazität der Zellen. Die größeren Dimensionen der 21700-Zellen ermöglichen eine höhere Kapazität pro Zelle im Vergleich zu den kleineren 18650-Zellen, was zu einer potenziell höheren Energiedichte führt und somit die Gesamtkapazität und Reichweite der Batterie positiv beeinflussen kann.\\

Ein weiterer wichtiger Aspekt ist die Energieeffizienz und Wärmeentwicklung. Die größere Oberfläche der 21700-Zellen trägt zu einer effizienteren Wärmeableitung bei, was insbesondere in Hochleistungsanwendungen wie E-Bikes von Vorteil ist. Eine verbesserte Wärmeableitung kann die Lebensdauer der Batterie verlängern und eine konstante Leistung unter anspruchsvollen Bedingungen gewährleisten.\\

Zusätzlich sollten technologische Fortschritte berücksichtigt werden. Die spätere Einführung der 21700-Zellen ermöglichte möglicherweise die Integration neuerer Technologien, die sich positiv auf Leistung, Sicherheit und Zuverlässigkeit auswirken. Obwohl 18650-Zellen etabliert und bewährt sind, könnten die neueren 21700-Zellen von Fortschritten in der Batterietechnologie profitieren.\\

Insgesamt legt die wissenschaftliche Analyse nahe, dass die Verwendung von 21700-Zellen aufgrund ihrer potenziell höheren Kapazität, verbesserten Wärmeableitung und der Möglichkeit, von technologischen Fortschritten zu profitieren, für E-Bike-Batterien vorteilhaft sein könnte.\\


\subsection{Wahl der Marke}
Die beiden Zelltypen, Samsung und Lishen, unterscheiden sich in mehreren Schlüsselparametern, die bei der Entscheidung für die Batteriekonstruktion eines E-Bikes berücksichtigt werden sollten.\\

Die Samsung-Zellen zeichnen sich durch eine höhere Kapazität von 4900 mAh aus, was potenziell zu einer längeren Betriebsdauer des E-Bikes führen könnte. Allerdings liegt der maximale Entladestrom bei 9,8A pro Zelle, was die Leistungsfähigkeit des Motors beeinflussen kann. \\

Im Gegensatz dazu bietet die Lishen-Zelle eine Kapazität von 4000 mAh, was etwas niedriger ist als bei Samsung. Allerdings ermöglicht sie einen höheren maximalen Entladestrom von 12A pro Zelle. Ein zusätzlicher Vorteil könnte sein, dass die Lishen-Zellen preiswerter sind, was eine wirtschaftliche Alternative darstellen könnte.\\

In einer Batteriekonfiguration für E-Bikes spielt die Anzahl der parallel geschalteten Zellen eine entscheidende Rolle für die Entladeleistung. Wenn weniger Zellen parallel geschaltet sind, bedeutet dies, dass die Gesamtkapazität der Batterie reduziert ist. Um dennoch die erforderliche Leistung bereitzustellen, muss der Entladestrom pro Zelle erhöht werden. Es gibt auch Zellen die einen Entladestrom von 35 Ampher liefern.

Die Entscheidung zwischen diesen Zellen hängt von den spezifischen Anforderungen des E-Bike-Projekts ab. Wenn eine längere Reichweite und eine höhere Kapazität priorisiert werden, könnten die Samsung-Zellen die bessere Wahl sein. Wenn jedoch eine höhere Leistungsfähigkeit des Motors und ein wirtschaftlicher Preis im Vordergrund stehen, könnte die Lishen-Zelle die geeignetere Option sein. Es ist ratsam, die Projektspezifikationen und das Budget sorgfältig zu prüfen, um die optimale Zellwahl zu treffen.\\

Es wurde sich für die Lishen-Zellen entschieden, da sie am billigsten waren und auch mehrer Parallel geschaltet werden um eine erhöhte Kapazität zuerreichen einher geht auch das genug leistung von der Batterie geleistet werden kann. Es werden 13 in Reihegeschaltet um 48 Volt zuerreichen, 5 parallel für eine Kapazität von 20 Ampherstunden und einen maximalen Entladestrom von 60 Ampher. Hier ist die Berechnung: \ref*{eq:Strom}. Die Begründung warum genau fünf parallel geschaltet werden findet man hier:\ref{introduction}\\

\begin{align}
    A_{\textrm{Battery}} =& 12 A\cdot 5 = 60 A\\
    V_{\textrm{Battery}} =& 3.6V \cdot 13 = 46.8V\\
    \textrm{Kapazität} =& 4000mAh \cdot 5 = 20 Ah
    \label{eq:Strom}
\end{align}

\section{Auswahl und Begründung eines Balancing-Battery Management Systems}
%Was ist ein BMS?
Ein entscheidender Aspekt bei der Konstruktion von Lithium-Ionen-Batterien für E-Bikes ist die Wahl des richtigen Battery Management Systems (BMS). Dieser Abschnitt erläutert die Funktionen, den Auswahlprozess und die Entscheidungsgrundlagen für die Verwendung eines Balancing-BMS anstelle eines BMS, das lediglich Überwachungsfunktionen bietet.\\
%Wie funkitoniert ein BMS?
Ein BMS ist ein elektronisches System, das die Leistung und Sicherheit von Lithium-Ionen-Batterien überwacht und regelt. In seiner Grundfunktion überwacht es Parameter wie Spannung, Strom, Temperatur und Ladezustand. Zudem bietet es Schutzmechanismen, um Überladung, Tiefentladung und Überhitzung zu verhindern.\\
%balanced VS only monetoring
Es gibt zwei Haupttypen von BMS: solche, die lediglich Überwachungsfunktionen bereitstellen, und solche, die auch ein sogenanntes Balancing implementieren. Balancing bedeutet, dass das BMS aktiv eingreift, um sicherzustellen, dass alle Zellen in der Batterie während des Lade- und Entladevorgangs ähnliche Spannungen aufweisen.\\

%Zweck und Vorteile eines Balancing-BMS:
Die Entscheidung für ein Balancing-BMS basiert auf der Notwendigkeit, eine gleichmäßige Verteilung der Spannung über alle Zellen hinweg sicherzustellen. Dieser Prozess ist entscheidend, um die Lebensdauer der Batterie zu verlängern und eine optimale Leistung zu gewährleisten. Im Gegensatz dazu konzentriert sich ein BMS mit ausschließlich Überwachungsfunktionen darauf, die Parameter zu überwachen, ohne aktiv in den Ladungsausgleich zwischen den Zellen einzugreifen.\\
%Warum ist es nicht gut wenn die Spannung der Zellen variert?
Die Variation der Spannung zwischen den Zellen in einer Batterie ist problematisch, da sie zu mehreren negativen Effekten führen kann, die die Leistungsfähigkeit, Sicherheit und Lebensdauer der Batterie beeinträchtigen.\\

Wenn die Spannung zwischen den Zellen variiert, kann dies zu einer ungleichmäßigen Entladung führen. Einige Zellen können schneller entladen werden als andere, was zu einem Ungleichgewicht in der Kapazitätsnutzung führt. Dies führt zu einer verkürzten Laufzeit und einer ineffizienten Nutzung der gesamten Batteriekapazität.\\

Eine Variation der Spannung birgt das Risiko von Überladung für einzelne Zellen. Wenn eine Zelle eine höhere Spannung aufweist als andere, besteht die Gefahr, dass sie über ihre Nennspannung hinausgeladen wird. Überladung kann zu einer thermischen Instabilität führen, was wiederum zu Sicherheitsrisiken wie Überhitzung, Bränden oder sogar Explosionen führen kann.\\

Umgekehrt kann eine zu niedrige Spannung in einer einzelnen Zelle bei Entladung auftreten. Dies kann zu einer Tiefentladung führen, was die Lebensdauer der Zelle drastisch verkürzt und zu irreparablen Schäden führen kann.\\

Ungleichmäßige Spannungsverteilung kann zu einer ungleichen Alterung der Zellen führen. Einige Zellen altern schneller als andere, was zu einem vorzeitigen Versagen der Batterie führen kann. Eine ausgeglichene Spannungsverteilung trägt dazu bei, die Lebensdauer der gesamten Batterie zu maximieren.\\

Eine ungleichmäßige Spannungsverteilung kann die Leistungsfähigkeit des Gesamtsystems beeinträchtigen. Insbesondere bei elektrischen Anwendungen wie E-Bikes kann dies zu einer unzureichenden Leistung und einer verringerten Reichweite führen.\\
%Begründung für die Entscheidung:
Die Wahl eines Balancing-BMS für die E-Bike-Batteriekonstruktion wurde aufgrund der Fokussierung auf eine maximal homogene Spannungsverteilung getroffen. Durch das aktive Balancing wird vermieden, dass einzelne Zellen aufgrund unterschiedlicher Lade- und Entladezyklen eine Ungleichgewichtssituation erfahren. Dies trägt nicht nur zur Verbesserung der Batterielebensdauer bei, sondern optimiert auch die gesamte Leistungsfähigkeit des E-Bike-Akkus.\\

Die Entscheidung für ein Balancing-BMS basierte auf der Zielsetzung, eine Batterie zu konstruieren, die nicht nur effizient und leistungsstark ist, sondern auch eine langfristige Zuverlässigkeit und Haltbarkeit gewährleistet.\\
%Wozu braucht man ein BMS?
%Wovor schützt ein BMS?
%Für welches BMS wurde ich entschieden?

%Wie schließt man ein BMS an?
%Wie sieht der schaltplan aus?
siehe schaltplan der gezeichnet wurde
und der schaltplan auf der Webseite


\section{Verbindung zwischen den Zählen}


\subsection{Löten vs. Punktscheißen}
Löten 
-schnell
-nicht stabil
-es wurde ein test gemacht hier bild ein blenden
-wenn nicht stabil hocher winderstand
-Hitze auf den Batterien genau gefahren kann in die Luft gehen
-Sicherung doch dünne drähte
-mehr platz für 
-Versuchsbeschreibung in einer Wissenschaftlichen Arbeit

Punktzuschweißen
-standard
-sehr stabil
-geringer Hitze auf den Batterien
-

Meschanisch verbinden
-am besten und am sicherseten
-kosten intensive
-schwer viel platz wird gebraucht
\section{Punktschweißgerät}
%Wie funktioniert schweißen?
%wie bewegen sich elektronen
%Welcher strom wird gebraucht?
%Man braucht einen Transformator?
%wie funktioniert einen transformator?

%Wo bekommt man einen transformator her?

%Wie muss man den transformator verändern?

%Wie kann man den Sekundären Block sicher entfernen?

%Weitere Probleme?

%Versuchs beschreibung in einer Wissenschaftlichen arbeit?
%Neuer Transformator?

%Wie muss der Transformator gewickelt werden? 

%Wie funktioniert die Transformator formel?

%Wie schaltet man das Punktschweißgerät ein?
%Controller
%schalter
%Taster

\section{Konstruktion der Batterie}

\subsection{Wahl des Anschlusses}
