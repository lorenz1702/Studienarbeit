 \chapter{Fazit}

Für das Design und die Implementierung eines voll funktionierenden E-Bikes gab es viele Herausforderungen, die zu Beginn der Studienarbeit nicht ersichtlich waren.
Neben dem elektrischen Gesamtdesign mussten viele handwerkliche Tätigkeiten erlernt werden.
Löten, Sägen, Feilen, Montieren und das Herstellen von elektrischen Verbindungen waren elementar für die erfolgreiche Implementierung des Fahrzeugs.
Ein entscheidender Faktor war dabei die Qualität der einzelnen Arbeitsschritte.
Eine kalte Lötstelle führte beispielsweise zu einem Teilausfall der Motorsteuerung und musste aufwändig gesucht und behoben werden.\\

Beim Gesamtdesign waren die Batteriegröße, die Leistung des Motors und die Auswahl des Fahrrads die Kernelemente des Designs.

Zudem konnten wichtige Erkenntnisse bezüglich des Fahrgefühls und Ideen auf zukünftige Entwicklungen gewonnen werden.
Außerdem einige andere Aspekte, wie die gewonnenen Erkenntnisse, das Fahrgefühl und einen Ausblick auf zukünftige Entwicklungen.\\


Die Größe der Batterie wurde so gewählt, dass sie einen großen Motor über eine längere Reichweite versorgen kann.
Die Reichweite des Fahrrads wird auf etwa 40 Kilometer bei maximaler Leistung geschätzt und noch mit Treten des Fahrers.
Bei 35 km/h und gleichzeitiges Treten des Fahrers, liegt die Reichweite bei 100 Kilometern.
Da jetzt die Fähigkeit gewonnen wurde, flexible Batterien zu bauen, mit dem Nebenprodukt der Arbeit, dem Punktschweißgerät.
Ist es sogar sinnvoll, mehrere kleine Batterien mitzuführen, um je nach Tour flexibel zu sein.\\



Die Wahl des Motors mit 1500 Watt war möglicherweise etwas überdimensioniert.
Dennoch war es wichtig, Erkenntnisse zu gewinnen, insbesondere bezüglich möglicher Probleme bei stärkeren Motoren, die bei schwächeren Modellen erst später auftreten würden.\\


Im Folgenden wird die Auswahl des Fahrrads reflektiert.
Das aktuell genutzte Fahrrad wurde aus Budgetgründen gewählt.
Es wäre jedoch sinnvoll, in Zukunft ein Fahrrad zu wählen, das über eine Vorder- und Hinterradfederung, oder zumindest über eine Vorderradfederung verfügt, um ein stabileres Fahrgefühl zu gewährleisten.
Obwohl die aktuellen Bremsen gut funktionieren, würden Scheibenbremsen vermutlich noch effektiver sein als die herkömmlichen Felgenbremsen.\\


Das Fahrgefühl des aktuellen Fahrrads ähnelt eher dem eines E-Rollers als dem eines herkömmlichen E-Bikes.
Dies liegt zum einen an der hohen Geschwindigkeit und zum anderen daran, dass das Fahrrad derzeit mit einem Gashebel gesteuert wird anstelle eines Pedalassistenz-Sensors.
Es sei jedoch angemerkt, dass der Verfasser generell das Fahrgefühl von E-Bikes nicht besonders genießt, da es kein direktes Feedback zwischen der aufgebrachten eigenen Kraft und der Endgeschwindigkeit gibt.\\

 \section{Ausblick}


Die gewonnenen Erkenntnisse dieser Studienarbeit können Grundlagen für zukünftige Projekte sein.
Eine bedeutende Errungenschaft besteht darin, dass ein Punktschweißgerät entwickelt wurde.
Dies ermöglicht die Herstellung weiterer Batterien in kleinerer Größe, wodurch es möglich wäre, mehrere Batterien für individuelle Touren anzupassen und gleichzeitig das Gewicht des Fahrrads zu reduzieren.\\

Des Weiteren würde die Verwendung eines schwächeren Motors, um eine angenehme Geschwindigkeit bei geringem Gewicht zu erreichen, etwa bis zu 40 oder 35 km/h bei eigener Tretunterstützung.
Die Position des Motors würde ans Vorderrad gelegt werden, um eine einfache Montage zu gewährleisten, ohne dass die Schaltung des Fahrrads angepasst werden muss.
Aufgrund der schwächeren Leistung des Motors würde es auch nicht zum Durchdrehen des Vorderrades kommen.
Aufgrund des geringeren Gewichts ist eine gleichmäßige Verteilung weniger entscheidend.\\


Durch die Kombination einer leichten 250-Watt-Vorderradmotor-Einheit, einem leichten Akku und einem Controller, der an den Flaschenhalterungen eines Fahrrads montiert werden kann, würde ein äußerst modulares Bausatzkonzept entstehen.
Dies ermöglicht die Umrüstung jedes Fahrrads in ein E-Bike ohne großen Aufwand.\\


Das langfristige Ziel des Autors ist es, das Bauen von E-Bikes zu erleichtern und zugänglicher zu machen.
Mit dem modularen Bausatzkonzept können verschiedene neue E-Bike Modelle entstehen und somit kann die Intermodalität in den Städten weiter vorangetrieben werden.













