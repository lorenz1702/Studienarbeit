
\chapter{Steuergerät programmieren}
Ein unverzichtbarer Bestandteil des E-Bike-Systems ist das Steuergerät, das die verschiedenen Betriebsmodi des Controllers steuert. Dieses Steuergerät ermöglicht es dem Fahrer, die Einstellungen und Leistungsparameter des E-Bikes je nach Bedarf anzupassen.\\

Die Umsetzung des Steuergeräts kann auf verschiedene Arten erfolgen:\\

    1. Einfacher Schalter: Eine Möglichkeit besteht darin, einen einfachen Schalter zu verwenden, der dem Fahrer die Wahl zwischen verschiedenen Modi bietet. Dieser Schalter kann beispielsweise zwischen den Modi für unterschiedliche Geschwindigkeitsstufen oder Leistungsniveaus umschalten.\\

    2. Anpassbares Display: Alternativ kann ein anpassbares Display integriert werden, das dem Fahrer eine visuelle Schnittstelle zur Steuerung des E-Bikes bietet. Mit einem solchen Display kann der Fahrer nicht nur zwischen verschiedenen Modi wählen, sondern auch Informationen wie Geschwindigkeit, Batterieladung und Fahrstrecke anzeigen.\\

Die Wahl zwischen einem einfachen Schalter und einem anpassbaren Display hängt von den individuellen Anforderungen und Präferenzen ab. Ein einfacher Schalter bietet eine unkomplizierte Bedienung, während ein Display zusätzliche Informationen und Anpassungsoptionen bietet.\\

Das Steuergerät spielt eine entscheidende Rolle dabei, die E-Bike-Fahrerfahrung anzupassen und zu personalisieren. In den folgenden Abschnitten werden wir uns mit weiteren Aspekten des E-Bike-Baus beschäftigen, einschließlich der Anpassung der Gangschaltung und der Vorbereitung des Fahrrads für die erhöhte Kraftauswirkung der elektrischen Unterstützung.\\
%Der Kontroller braucht auch ein Steuergeraet welchhe die verschiedenen Modi umschaltet. Dies kann durch einen einfachen Schalter umgesetzt werden oder eine Display kann angepasst werden.